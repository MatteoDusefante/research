\documentclass{article}

\usepackage[a4paper,total={6in, 9in}]{geometry}

\title{Statement of Purpose}
\author{Matteo Dusefante}
\date{Copenhagen, \today}

\newcommand{\CC}{C\nolinebreak\hspace{-.05em}\raisebox{.4ex}{\tiny\bf +}\nolinebreak\hspace{-.10em}\raisebox{.4ex}{\tiny\bf +}}


\begin{document}
\maketitle

I earned my Bachelor of Science in Computer Science in 2010, defending a dissertation that discusses several complexity aspects in multi agent systems. The dissertation, entitled ``Complexity Analysis in Multi Agent Systems'', deals with the main theoretical aspects of the multi agent systems and some related complexity issues. While working on the thesis, I improved both my research skills and my theoretical background of algorithms and complexity. Besides this, during the three-year degree I took the foundation courses of information theory, algorithms and data structures.

\bigskip

During my Master of Science studies in computer science, I focused on advanced algorithms and complexity. More specifically, I took several courses aimed at improving my theoretical knowledge. The courses I enrolled to range from algorithms on strings and sequences to numerical algorithms. What is more, I completed my general knowledge of the subject by attending courses and laboratories covering logic programming, operations research and bioinformatics topics.

My fields of interest include

\begin{itemize}
\item \emph{Algorithms on strings and sequences, advanced algorithms, randomized algorithms}: String matching algorithms and algorithms on trees; Advanced data structures for patterns manipulation; Montecarlo and Las Vegas algorithms.
\item \emph{Complexity analysis over computational models and algorithms}: Abstract computational models; Computational classes, computational reductions and completeness; Bisimulation Theory.
\item \emph{Numerical algorithms and High Performance Computing}: Non linear equations and linear systems; Sparse Matrix Technologies; Approximation of data and functions, Fourier Transform; Differential equations and dynamical systems, Ordinary Differential Equation and Delay Differential Equation.
\item \emph{Logic programming and constraint programming}: Local search and Integer Linear Programming; Constraint Satisfaction and Optimization Problems; Constraint Programming and Constraint Logic Programming over Finite Domains; Answer Set Programming.
\end{itemize}

Besides the aforementioned theoretical subjects, during my whole university studies, I enforced my programming skills by developing several softwares. This led me to a good proficiency in a wide set of programming techniques and programming languages. In more detail, my programming languages of choice are Java and \CC/C. I developed in addition a number of tools in others programming languages, such as Python, Prolog, Haskell, Curry, Matlab.

\bigskip

My master thesis, entitled ``An integrated system for protein prediction'', is concerned with the development of an integrated system aimed at efficiently solving the protein structure prediction problem. More specifically, the dissertation provides a graphical user interface written in Java which in turn handles two constraint solvers written in \CC. While working on my master thesis I had the opportunity to work in a research team formed by several professors and Ph.D. students working at different Universities. I collaborated mainly with the Ph.D. student Federico Campeotto from the New Mexico State University and with prof. Alessandro Dal Palù from the University of Parma. My efforts culminated in the early release of the system being presented at the 10th Workshop on Constraint-Based Methods for Bioinformatics, in Lyon, France, on September 8th, 2014

\bigskip

During my graduate studies I intend to continue improving my theoretical knowledge and my programming skills. I hope I will be working in a team formed by scientists of heterogeneous backgrounds and will have the chance to face and possibly solve cutting edge algorithmic problems. I am aware my current skills have to be improved, although they cover the foundations of algorithms. Since many applicative problems tend to be multidisciplinary, I firmly believe that a graduate program should also improve the interpersonal skills of the researcher. To pursue all these objectives, I am willing to undertake my Ph.D. studies in an international program. The specifics of my future research are obviously undefined, but as a long term objective I can mention working in the field of advanced algorithms.

\end{document}

